\graphicspath{{img/algor/}}
\chapter{Codigos}


\subsection{Detección de las raices de una función en un intervalo dado}

\begin{listing}[H]
    \begin{minted}[mathescape,
               gobble=8,
               frame=lines,
               framesep=2mm]{python}
        from numpy import zeros
        
        def bracketing(fun, a, b, N):
            msg = "Maximum number of iterations reached."
            dx = (b - a)/(N - 1)
            iroot = 0
            x2 = a
            xR = zeros(N, float)
            for i in range(0, N):
                x1 = x2
                x2 = x1 + dx
                if (fun(x1) * fun(x2)) < 0:
                    msg = "A change of sign was found."
                    iroot = iroot + 1
                    xR[i] = x1
            return xR, msg

        # Function call
        a = -10.0
        b = 10.0
        N = 21
        fun = lambda x: x**3 + 4*x**2 - 10
        xR, msg = bracketing(fun, a, b, N)
        print(msg)
        print(xR)
    \end{minted}
    \caption{Método de búsquedas incrementales en Python.}
    \label{lst:bracketing}
\end{listing}




\subsection{Método de bisección para la localización de raices en un intervalo dado}

\begin{listing}[H]
    \begin{minted}[mathescape,
               gobble=8,
               frame=lines,
               framesep=2mm]{python}
        def bisection(fun, a, b, xtol=1e-6, ftol=1e-12, verbose=False):
            """
            Use bisection method to estimate the root of a real function
            """
            if fun(a) * fun(b) > 0:
                c = None
                msg = "The function should change sign in the interval."
            else:
                nmax = int(ceil(log2((b - a)/xtol)))
                for cont in range(nmax):
                    c = 0.5*(a + b)
                    if verbose:
                        print("n: {}, x: {}".format(cont, c))
                    if abs(fun(c)) < ftol:
                        msg = "Root found with desired accuracy."
                        break
                    elif fun(a) * fun(c) < 0:
                        b = c
                    elif fun(b) * fun(c) < 0:
                        a = c
                    msg = "Maximum number of iterations reached."
            return c, msg
        
        x, msg = bisection(lambda x: x**3 + 4*x**2 -10, -2, 2, xtol=1e-4,
                           verbose=True)
        print(msg)
        print(x)
    \end{minted}
    \caption{Método de bisección en Python.}
    \label{lst:bisection}
\end{listing}


\subsection{Método de Newton-Raphson para la localización de raices en un intervalo dado}

\begin{listing}[H]
    \begin{minted}[mathescape,
               gobble=8,
               frame=lines,
               framesep=2mm]{python}
        def newton(fun, grad, x, niter=50, ftol=1e-8, verbose=False):
            """
            Use Newton method to estimate the root of a real function
            """
            msg = "Maximum number of iterations reached."
            for cont in range(niter):
                if abs(grad(x)) < ftol:
                    x = None
                    msg = "Derivative near to zero."
                    break
                if verbose:
                    print("n: {}, x: {}".format(cont, x))
                x = x - fun(x)/grad(x)
                if abs(fun(x)) < ftol:
                    msg = "Root found with desired accuracy."
                    break
            return x, msg
        
        func = lambda x:x**3 + 4*x**2 - 10
        deriv = lambda x: 3*x**2 + 8*x
        result = newton(func, deriv, 2, verbose=True)
        print(result)
    \end{minted}
    \caption{Método de Newton-Rapshon en Python.}
    \label{lst:newton-raphson}
\end{listing}


\subsection{Integración numérica: Regla del trapecio}

\begin{minted}[mathescape,
           gobble=4,
           frame=lines,
           framesep=2mm]{python}
    import numpy as np
    from sympy import symbols, integrate


    def trapz(fun, x0, x1, n):
    """Trapezoidal rule for integration
    
    Parameters
    ----------
    fun : callable
         Function to integrate.
    x0 : float
         Initial point for the integration interval.
    x1 : float
         End point for the integration interval.
    n : int
         Number of points to take in the interval.
    
    Returns
    -------
    inte : float
         Approximation of the integral
    
    """
    x = np.linspace(x0, x1, n)
    y = fun(x)
    dx = x[1] - x[0]
    inte = 0.5*dx*(y[0] + y[-1])
    for cont in range(1, n - 1):
    inte = inte + dx*y[cont]
    return inte


    fun = lambda x: x**3 + 4*x**2 - 10
    for cont in range(2, 11):
        numeric_int = trapz(fun, -1, 1, cont)
        print("Approximation for {} subdivisions: {:.6f}".format(cont - 1,
               numeric_int))

    x  = symbols('x')
    analytic_int = integrate(fun(x) , (x , -1 , 1))
    print("Analytic integral: {:.6f}".format(float(analytic_int)))
\end{minted}

\subsection{Integración numériva: Cuadratura Gaussiana}

\begin{minted}[mathescape,
	gobble=4,
	frame=lines,
	framesep=2mm]{python} 
    import numpy as np
    from scipy.special import roots_legendre # Gauss points and weights
    from sympy import symbols, integrate
    
    
    def gauss1d(fun, x0, x1, n):
    """Gauss quadrature in 1D
    
    Parameters
    ----------
    fun : callable
        Function to integrate.
    x0 : float
        Initial point for the integration interval.
    x1 : float
        End point for the integration interval.
    n : int
        Number of points to take in the interval.
    
    Returns
    -------
    inte : float
         Approximation of the integral
    
    """
    xi, wi = roots_legendre(n)
    inte = 0
    h = 0.5 * (x1 - x0)
    xm = 0.5 * (x0 + x1)
    for cont in range(n):
        inte = inte + h * fun(h * xi[cont] + xm) * wi[cont]
    return inte

    fun = lambda x: x**3 + 4*x**2 - 10
    gauss_inte = gauss1d(fun, -1, 1, 4)
    x  = symbols('x')
    analytic_inte = integrate(fun(x) , (x , -1 , 1))
    print("Analytic integral: {:.6f}".format(float(analytic_inte)))
    print("Gauss quadrature: {:.6f}".format(gauss_inte))
\end{minted}


\subsection{Integración en un 2 dimensiones}

\begin{mdframed}[linecolor=black,
	topline=true,
	bottomline=true,
	leftline=false,
	rightline=false]
	\begin{minted}[mathescape,
	gobble=8,
	framesep=2mm]{python}
	import numpy as np
	from scipy.special import roots_legendre # Gauss points and weights
	
	
	def gauss2d(fun, x0, x1, y0, y1, nx, ny):
	"""Gauss quadrature for a rectangle in 2D
	
	Parameters
	----------
	fun : callable
	Function to integrate.
	x0 : float
	Initial point for the integration interval in x.
	x1 : float
	End point for the integration interval in x.
	y0 : float
	Initial point for the integration interval in y.
	y1 : float
	End point for the integration interval in y.
	nx : int
	Number of points to take in the interval in x.
	ny : int
	Number of points to take in the interval in y.
	
	Returns
	-------
	inte : float
	Approximation of the integral
	
	"""
	xi, wi = roots_legendre(nx)
	yj, wj = roots_legendre(ny)
	inte = 0
	hx = 0.5 * (x1 - x0)
	hy = 0.5 * (y1 - y0)
	xm = 0.5 * (x0 + x1)
	ym = 0.5 * (y0 + y1)
	for cont_x in range(nx):
	for cont_y in range(ny):
	f = fun(hx * xi[cont_x] + xm, hy * yj[cont_y] + ym)
	inte = inte + hx* hy * f * wi[cont_x] * wj[cont_y]
	return inte
	
	
	fun = lambda x, y: 3*x*y**2 - x**3
	gauss_inte = gauss2d(fun, 0, 2, 0, 2, 2, 2)
	print("Gauss quadrature: {:.6f}".format(gauss_inte))
	\end{minted}
\end{mdframed}






